\chapter*{Введение}
\addcontentsline{toc}{chapter}{Введение}

В настоящее время существует огромное количество переносимых и стационарных устройств, общение между которыми происходит посредством спутниковой связи. Ввиду различных факторов сигналы, приходящие с искусственных спутников Земли, имеют шумы. Одним из методов их устранения является цифровая фильтрация \cite{intro}.

Понятие "фильтрация" широко используется во многих областях. Под этим термином обычно понимают создание препятствий для прохождения каких-либо объектов. Например, в электротехнике фильтры используются для преобразования электрических сигналов из одной формы в другую, главным образом, чтобы исключить (отфильтровать) различные частоты в сигнале \cite{aft}.

В теории цифровой обработки сигналов понятие фильтрации используется в  более широком смысле. Пусть в результате дискретизации некоторого непрерывного сигнала получена цифровая последовательность данных. Будем понимать под цифровой фильтрацией такое преобразование сигнала, при котором связь между  входной  и выходной последовательностями является линейной.


Цель данной научно-исследовательской работы --- провести обзор существующих алгоритмов цифровой фильтрации сигналов, приходящих с искусственных спутников Земли, выполнив их сравнение.

Для достижения указанной выше цели следует выполнить задачи:
\begin{itemize}
	\item изучить принципы работы цифровых фильтров;
	\item выполнить классификацию шумов по природе их возникновения и качественным характеристикам;
	\item изучить существующие методы фильтрации сигналов;
	\item выполнить сравнительный анализ предложенных алгоритмов.
\end{itemize}


