\chapter*{Заключение}
\addcontentsline{toc}{chapter}{Заключение}

В ходе выполнения данной работы были изучены принципы работы цифровых фильтров, выполнена классификация шумов по природе их возникновения и качественным характеристикам, изучены существующие методы фильтрации сигналов, а также выполнен сравнительный анализ предложенных алгоритмов.

На основе проделанной работы можно сделать вывод о том, что не существует одного универсального метода фильтрации, который можно применять в любой момент. Каждый из методов имеет свои достоинства и недостатки при фильтрации конкретных сигналов. Однако стоит отметить, что метод медианной фильтрации является наиболее предпочтительным в большинстве ситуаций ввиду своей низкой вычислительной сложности и высокого качества фильтрации. Также метод медианной фильтрации возможно сочетать с другими видами фильтраций для достижения наилучшего результата.	
